\documentclass[a4paper,12pt]{exam}

\usepackage{fontspec}

\setromanfont{Arial}
\setsansfont{Arial}
\setmonofont{Arial}

\usepackage{minted}
\usemintedstyle{trac}
\usepackage{csquotes}
\usepackage{xcolor}


\firstpageheader{EE3731C}{Programming Assignment}{\today}
\runningheader{EE3731C}{Programming Assignment (Continued)}{\today}
\footer{}{\thepage}{}

\begin{document}
\begin{questions}

% 1
\question 

\begin{parts} 

% 1 (a)
\part

\mint[breaklines]{matlab}|NumericArray = double('Pallas cats have longest fur')| 
\begin{minted}[breaklines]{text}

NumericArray =

  Columns 1 through 18

    80    97   108   108    97   115    32    99    97   116   115    32   104    97   118   101    32   108

  Columns 19 through 28

   111   110   103   101   115   116    32   102   117   114

\end{minted}

\mint[breaklines]{matlab}|CharacterArray = char(NumericArray)| 
\begin{minted}[breaklines]{text}

CharacterArray =

    'Pallas cats have longest fur'

\end{minted}

\mint[breaklines]{matlab}|CharacterArray = char([71 117 101 115 115 32 119 104 97 116 32 105 116 32 105 115])| 
\begin{minted}[breaklines]{text}

CharacterArray =

    'Guess what it is'

\end{minted}

% 1 (b)
\part
    
\mint[breaklines]{matlab}|NumericArray = double('Pallas cats have longest fur')| 
\begin{minted}[breaklines]{text}

ans =

  Columns 1 through 10

    16     1    12    12     1    19    27     3     1    20

  Columns 11 through 20

    19    27     8     1    22     5    27    12    15    14

  Columns 21 through 28

     7     5    19    20    27     6    21    18

\end{minted}

% 1 (c)
\part
    
\mint[breaklines]{matlab}|double2char([18 5 13 5 13 2 5 18 27 25 15 21 18 27 13 1 19 11])| 
\begin{minted}[breaklines]{text}

ans =

    'remember your mask'

\end{minted}

\end{parts}

% 2
\question

\begin{parts}
% 2(a)
\part
\begin{minted}[breaklines]{text}

frank_encrypted_txt =

    'urnnrtqsritqyrplqnxc ltiqvkz'

\end{minted}

% 2(b)
\part
\begin{minted}[breaklines]{text}

frank_decrypted_txt =

    'this is a secret'

\end{minted}
\end{parts}

% 3
\question

\begin{parts}
% 3 (a)
\part

\mint[breaklines]{matlab}|pr_trans(1, 1)| 
\begin{minted}[breaklines]{text}

ans =

   9.8020e-05

\end{minted}

\mint[breaklines]{matlab}|pr_trans(2, 3)| 
\begin{minted}[breaklines]{text}

ans =

   4.9505e-04

\end{minted}

The highest probability in \verb|pr_trans| is 0.7920.
This occurs at \verb|pr_trans(17,21)|, therefore this corresponds to the transition from the 17th letter of the alphabet, Q, to the 21st letter of the alphabet, U.

% 3 (b)
\part

\mint[breaklines]{matlab}|logn_pr = logn_pr_txt(frank_encrypted_txt, pr_trans)| 
\begin{minted}[breaklines]{text}

logn_pr =

  -5.6219e+03

\end{minted}

\mint[breaklines]{matlab}|logn_pr = logn_pr_txt(frank_original_txt, pr_trans)| 
\begin{minted}[breaklines]{text}

logn_pr =

  -2.3339e+03

\end{minted}

% 3 (c)
\part

\begin{verbatim}p(frank_encrypted_txt | frank_decrypt_key)\end{verbatim}
\begin{minted}[breaklines]{text}

logn_pr =

  -2.3339e+03

\end{minted}

\begin{verbatim}p(frank_encrypted_txt | mystery_decrypt_key)\end{verbatim}
\begin{minted}[breaklines]{text}

logn_pr =

  -6.0547e+03

\end{minted}

\end{parts}

% 4
\question

\begin{parts}
% 4 (a)
\part
\begin{subparts}

% 4 (a) (i)
\subpart
\mint[breaklines]{matlab}|[accept_new_key, prob_accept] = metropolis(frank_decrypt_key, mystery_decrypt_key, pr_trans, frank_encrypted_txt)|
\begin{minted}[breaklines]{text}

accept_new_key =

     0


prob_accept =

     0

\end{minted}

% 4 (a) (ii)
\subpart

\mint[breaklines]{matlab}|new_decrypt_key = frank_decrypt_key|

\mint[breaklines]{matlab}|new_decrypt_key([13 14]) = new_decrypt_key([14 13])|

\mint[breaklines]{matlab}|[accept_new_key, prob_accept] = metropolis(frank_decrypt_key, new_decrypt_key, pr_trans, frank_encrypted_txt)|
\begin{minted}[breaklines]{text}

accept_new_key =

     0


prob_accept =

   5.1586e-48

\end{minted}
\end{subparts}

% 4(b)
\part
\mint[breaklines]{matlab}|mcmc_decrypt_text(frank_encrypted_txt, pr_trans)|

Run 15000: log probability = -2334.5558

the deep grief which this scene had at first excited juickly gave way to  rage and despair  they were dead  and i lived  their murderer also lived   and to destroy him i must drag out my weary existence  i knelt on the grass  and kissed the earth and with juivering lips exclaimed     by the  sacred earth on which i kneel  by the shades that wander near me  by the  deep and eternal grief that i feel  i swear  and by thee  o night  and the  spirits that preside over thee  to pursue the d  mon who caused this misery   until he or i shall perish in mortal conflict  for this purpose i will  preserve my life  to execute this dear revenge will i again behold the sun  and tread the green herbage of earth  which otherwise should vanish from my  eyes for ever  and i call on you  spirits of the dead  and on you  wandering  ministers of vengeance  to aid and conduct me in my work  let the cursed  and hellish monster drink deep of agony  let him feel the despair that now  torments me      


The estimated \verb|decrypt_key| is \verb|'mbndziqytjuexlwv acspfkohrg'|

\begin{minted}[breaklines]{text}
decrypt_key =

  Columns 1 through 10

    13     2    14     4    26     9    17    25    20    10

  Columns 11 through 20

    21     5    24    12    23    22    27     1     3    19

  Columns 21 through 27

    16     6    11    15     8    18     7

\end{minted}

while \verb|frank_decrypt_key| is \verb|'mbndzijytquelxwv acspfkohrg'|

\begin{minted}[breaklines]{text}
frank_decrypt_key =

  Columns 1 through 10

    13     2    14     4    26     9    10    25    20    17

  Columns 11 through 20

    21     5    24    12    23    22    27     1     3    19

  Columns 21 through 27

    16     6    11    15     8    18     7

\end{minted}

The letters j (10) and q (17) in columns 7 (g) and 10 (j) are swapped. This results in the words \textquote{juickly} and \textquote{juivering}.

For \verb|decrypt_key|, \verb|log probability = -2334.5558|.

For \verb|frank_decrypt_key|, \verb|log probability = -2333.896|.

exp(ln P - ln P\textsubscript{frank}) = 0.52

For MCMC, we want to get a posterior so we have to sometimes accept moves in the other direction, even when ln P\textsubscript{new} < ln P\textsubscript{current}. No matter how many iterations the program is run or which letters are being swapped there is always a non-zero probability that a swap will be accepted.

% 4 (c)
\part
\mint[breaklines]{matlab}|mcmc_decrypt_text(mystery_encrypted_txt, pr_trans)|

Run 15000: log probability = -1620.1128

daytimes we paddled all over the island in the canoe  it was mighty cool  and shady in the deep woods  even if the sun was blaxing outside    we  went winding in and out amongst the trees  and sometimes the vines hung  so thick we had to back away and go some other way    well  on every old  broken down tree you could see rabbits and snakes and such things  and  when the island had been overflowed a day or two they got so tame  on  account of being hungry  that you could paddle right up and put your  hand on them if you wanted to  but not the snakes and turtles  they would  slide off in the water    the ridge our cavern was in was full of them   we could a had pets enough if we d wanted them

The estimated \verb|decrypt_key| is \verb|' zphkfimwedvunlsgqbjycrxota'|

\begin{minted}[breaklines]{text}
decrypt_key =

  Columns 1 through 10

    27    26    16     8    11     6     9    13    23     5

  Columns 11 through 20

     4    22    21    14    12    19     7    17     2    10

  Columns 21 through 27

    25     3    18    24    15    20     1

\end{minted}

while \verb|mystery_decrypt_key| is \verb|' xphkfimwedvunlsgqbjycrzota'|

\begin{minted}[breaklines]{text}
mystery_decrypt_key =

  Columns 1 through 10

    27    24    16     8    11     6     9    13    23     5

  Columns 11 through 20

     4    22    21    14    12    19     7    17     2    10

  Columns 21 through 27

    25     3    18    26    15    20     1

\end{minted}

The letters x (24) and z (26) in columns 2 (b) and 24 (x) are swapped. This results in the word \textquote{blaxing}.

For \verb|decrypt_key|, \verb|log probability = -1620.1128|.

For \verb|mystery_decrypt_key|, \verb|log probability = -1621.2173|.

exp(ln P - ln P\textsubscript{mystery}) = 3.02

Based on the training text, the decrypted text based on the \textquote{wrong} decryption key is more probable than the decrypted text based on the \textquote{correct} decryption key. 

As mentioned in the previous part of this question, even if the probability of a new sequence of characters is lower, there is still a chance for the swap to be accepted. But the probability in this case is much lower than 0.52, at only exp(ln P\textsubscript{mystery} - ln P) = 0.33. Among the printed iterations, the less probable x-z positions only occurred once at iteration 4000. In contrast, among the printed iterations, starting from iteration 4000, for probability of 0.52, 5 out of 12 times the less probable j-q positions occurred.
\end{parts}

% 5
\question
For the case in 5 (b), stop attempting to get samples from the posterior after a sufficient number of iterations. By then, it is likely that they have already been explored and we are unlikely to find a more probable sequence of characters through them.

For the case in 5 (c), we will have to make the decrypted text based on the \textquote{correct} decryption key more probable than the decrypted text based on the \textquote{wrong} decryption key.

The training text, \textit{The Cash Boy}, was written in 1887. The two encrypted texts, \textit{Frankenstein} and \textit{Adventures of Huckleberry Finn} were written in 1818 and 1884, respectively. The authors were American men except Mary Wollstonecraft, a British woman.

Therefore, for 5 (c), the texts were both written by American men, Horatio Alger and Mark Twain, in the 1880s. They were also born only three years apart. The difference must be something else.

Comparing \textit{The Cash Boy} with \textit{Adventures of Huckleberry Finn}, the most obvious difference is the amount of conversation in the former. While Twain's novel is \textquote{noted for its colorful description of people and places along the Mississippi River}, Alger wrote about \textquote{impoverished boys and their rise from humble backgrounds to lives of middle-class security and comfort through good works.} The training text simply contains too much conversation and not enough description for phrases like \textquote{the sun was blazing outside}.

I suggest using an 1880s novel written by American men born in the 1830s that includes more description than conversation, preferably a Great American Novel, such as \textquote{The Adventures of Tom Sawyer}.

The issue with this would be the fact that this is obviously overfitting.

For both cases, we can also use a dictionary to check the validity of a word. The dictionary should be of the same language and from the same era as the training text. However, this solution may wrongly consider proper nouns such as \textquote{Juickly} or \textquote{Blaxing} invalid, for example.
\end{questions}

\end{document}
